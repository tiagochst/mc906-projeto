\documentclass[10pt,a4paper]{article}
\usepackage[T1]{fontenc}
\usepackage[brazil]{babel}
\usepackage[utf8]{inputenc}


\usepackage{ae,aecompl}
\usepackage{pslatex}
\usepackage{epsfig}
\usepackage{geometry}
\usepackage{url}
\usepackage{textcomp}
\usepackage{ae}
\usepackage{subfig}
\usepackage{indentfirst}
\usepackage{textcomp}
\usepackage{color}
\usepackage{setspace}
\usepackage{verbatim}
%  ABACO -- Conjunto de macros para desenhar o 'abaco

%  Desenho original de Hans Liesenberg

%  Macros de Tomasz Kowaltowski

%  DCC -- IMECC -- UNICAMP

%  Mar,co de 1988  --  Vers~ao 1.0

% Ajustado para LaTeX da SUN -- Mar,co de 1991

% ---------------------------------------------------------

%  Chamada:   \ABACO{d1}{d2}{d3}{d4}{esc}
%             com:  di's -- os quatro d'igitos;
%	           esc  -- fator de escala

% ---------------------------------------------------------

%  DEFINI,C~OES AUXILIARES

% ---------------------------------------------------------


%  Forma o d'igito pequeno (0 ou 1)

\newcommand{\ABACODP}[1]{%
%
\thicklines
%    
\begin{picture}(8,0)
    \ifcase#1{   %  caso 0
       \put(0,0)    {\line(1,0){4}}
       \multiput(5,0)(2,0){2}{\oval(2,4)}}
    \or{         %  caso 1
       \put(2,0)    {\line(1,0){4}}
       \multiput(1,0)(6,0){2}{\oval(2,4)}}
    \fi
\end{picture}
    } % \ABACODP

% Forma o d'igito grande (0 a 4)

\newcommand{\ABACODG}[1]{%
%
\thicklines
%    
\begin{picture}(14,0)
    \ifcase#1{   % caso 0
       \multiput(1,0)(2,0){5}{\oval(2,4)}}
       \put(10,0)   {\line(1,0){4}}
    \or{         % caso 1
       \multiput(1,0)(2,0){4}{\oval(2,4)}}
       \put(8,0)   {\line(1,0){4}}
       \put(13,0)   {\oval(2,4)}
    \or{         % caso 2
       \multiput(1,0)(2,0){3}{\oval(2,4)}
       \put(6,0)   {\line(1,0){4}}
       \multiput(11,0)(2,0){2}{\oval(2,4)}}
    \or{         % caso 3
       \multiput(1,0)(2,0){2}{\oval(2,4)}
       \put(4,0)   {\line(1,0){4}}
       \multiput(9,0)(2,0){3}{\oval(2,4)}}
    \or{         % caso 4
       \put(1,0)  {\oval(2,4)}}
       \put(2,0)   {\line(1,0){4}}
       \multiput(7,0)(2,0){4}{\oval(2,4)}
    \fi
\end{picture}
    } % \ABACODG
       
% Forma um d'igito (0 a 9)

\newcommand{\ABACOD}[1]{%
%
    \ifnum#1>9
       \errmessage{#1: Argumento invalido para ABACO}
    \fi
    \ifnum#1<0
       \errmessage{#1: Argumento invalido para ABACO}
    \fi
%
\begin{picture}(24,0)
%    
    \ifnum#1<5
       \put(16,0) {\ABACODP{0}}
    \else   
       \put(16,0) {\ABACODP{1}}
    \fi
%    
    \ifnum#1<5
       \put(0,0)  {\ABACODG{#1}}
    \else
       \ifcase#1\or \or \or \or
          \or  \put(0,0)  {\ABACODG{0}}
          \or  \put(0,0)  {\ABACODG{1}}
          \or  \put(0,0)  {\ABACODG{2}}
          \or  \put(0,0)  {\ABACODG{3}}
          \or  \put(0,0)  {\ABACODG{4}}
       \fi
    \fi   
\end{picture}
    } % \ABACOD
    
% -------------------------------------------------

%  DEFINI,C~AO PRINCIPAL
    
\newcommand{\ABACO}[5]{%
    \setlength{\unitlength}{#5mm}
%
    \thinlines
%   
\begin{picture}(28,25)
%   
% moldura
%
% externa
%
        \put(0,0)            {\line(0,1){25}}
        \put(0,0)            {\line(1,0){28}}
        \put(28,0)           {\line(0,1){25}}
        \put(0,25)           {\line(1,0){28}}
% interna
        \put(2,2)            {\line(0,1){21}}
	\put(26,2)           {\line(0,1){21}}
	\put(16,2)           {\line(0,1){21}}
	\put(18,2)           {\line(0,1){21}}
	\put(2,2)            {\line(1,0){14}}
	\put(16,2)           {\line(1,-1){1}}
	\put(17,1)           {\line(1,1){1}}
	\put(18,2)           {\line(1,0){8}}
	\put(2,23)           {\line(1,0){14}}
	\put(16,23)          {\line(1,1){1}}
	\put(17,24)          {\line(1,-1){1}}
	\put(18,23)          {\line(1,0){8}}
	\put(0,0)            {\line(1,1){2}}
	\put(0,25)           {\line(1,-1){2}}
	\put(28,0)           {\line(-1,1){2}}
	\put(28,25)          {\line(-1,-1){2}}
%
%   
% d'igitos
%
%   
       \put(2,20)  {\ABACOD{#1}}
       \put(2,15)  {\ABACOD{#2}}
       \put(2,10)  {\ABACOD{#3}}
       \put(2,5)   {\ABACOD{#4}}
%      
\end{picture}
    } % \ABACO
    
 


 %\onehalfspacing
  \doublespacing
\begin{document}

% CAPA
  \thispagestyle{empty}
  
  \begin{minipage}[h]{0.10\linewidth}
    \ABACO{1}{9}{6}{9}{0.5} 
  \end{minipage}
  \begin{minipage}[h!]{0.7\linewidth}
    \vspace*{\fill}
    \centering
    {\large \textbf{UNIVERSIDADE~ESTADUAL~DE~CAMPINAS}}\\ 
    {\large INSTITUTO~DE~COMPUTAÇÃO}                   
    \vspace*{\fill} 
  \end{minipage}
    \\\vspace{0.5cm}
  
  \begin{center} 
    \rule{11.0cm}{0.4pt}\vspace*{-\baselineskip}\vspace{-2.0pt}
    \rule{11.0cm}{1.6pt} \vspace*{-\baselineskip}
      {\Large \textsc{Reconhecimento de faces em redes sociais}}\vspace{3.2pt}
    \rule{11.0cm}{0.4pt}\vspace*{-\baselineskip}\vspace{3.2pt} \rule{11.0cm}{1.6pt}\\
    {\textsl{Proposta de projeto}}
    \\\vspace{1cm}
    \begin{tabular}{ll}
      Carlos Eduardo Rosa Machado & \textbf{RA}: 059582\\
       \multicolumn{2}{c}{\small \emph ra059582@students.ic.unicamp.br}  \\
      Douglas Alves Germano        & \textbf{RA}: 060210\\
      \multicolumn{2}{c}{\small \emph ra060210@students.ic.unicamp.br } \\
      Tiago Chedraoui Silva        & \textbf{RA}: 082941\\
      \multicolumn{2}{c}{\small \emph ra082941@students.ic.unicamp.br} \\
    \end{tabular}
  \end{center}
  \vspace{0.5cm}
 \tableofcontents
\newpage 
  \section{Introdução e Motivação}

	Rede social é uma estrutura composta por indivíduos ou organizações, chamados de nós, que se conectam por um ou vários tipos de relações, tais como amizade, parentesco, interesses em comum, entre outras.
	
	O Facebook é uma rede social gratuita, fundada em 2004, e atualmente possui mais de 700 milhões de usuários ativos.

	Uma pessoa pode criar um perfil na rede social, tornando-se um
        usuário do sistema. Para a criação de um perfil, ela deve
        fornecer informações pessoais como nome, localização, endereço
        eletrônico, e outras que desejar. Entre as opções, o usuário
        pode fornecer uma foto para ser mostrada no perfil, que
        qualquer outro usuário terá acesso irrestrito e a
        possibilidade de visualizá-la. Ainda assim, esta é uma prática
        bastante utilizada pela maioria dos usuários, pois proporciona
        fácil reconhecimento pelas demais pessoas. Além disso, o
        usuário pode colocar outras fotos pessoais, para as quais
        controlará o acesso, deixando restrito, por exemplo, apenas para aqueles que fazem parte da sua rede de relacionamentos no sistema.

	Pelo próprio intuito da rede, um usuário cadastrado quer se conectar e se relacionar virtualmente com as pessoas que conhece, ou até mesmo conhecer pessoas novas, quando algum interesse mútuo ocorre.

	Desta forma, o Facebook tem algumas ferramentas à disposição dos usuários, que permitem pessoas serem encontradas por outras usando alguma informação, tal como nome, endereço eletrônico ou endereço físico.

	Por isso, a ambição do projeto está em criar uma nova
        ferramenta, um novo meio para que as pessoas se encontrem e se
        conheçam. Apesar do Facebook ter bons buscadores que encontram
        facilmente uma pessoa, precisamos sempre ter alguma informação
        da pessoa, que podemos não saber e não conseguir obter
        rapidamente. Contudo, podemos ter uma foto, e é neste caso que
        entra a nossa proposta de solução: buscar um perfil dado uma
        foto.
\newpage
  \section{O problema}
É muito comum a situação em que uma pessoa tenta procurar alguém em redes sociais mas não conhece ou não lembra dados importantes sobre o indivíduo, como o seu nome, o seu endereço eletrônico ou o seu local de trabalho.

	Entretanto, em alguns casos, a pessoa pode ter em mãos uma foto tirada com a outra em um seminário, um evento ou uma festa.

	Utilizando essa foto para auxiliar a busca pela página do perfil desejado, criou-se um sistema (e eventualmente um aplicativo) que, dado uma foto de entrada contendo uma face a ser procurada, retorna os perfis no Facebook com os rostos mais semelhantes ao objetivo dentre os amigos de amigos do usuário.

	Inicialmente construiu-se um acervo de mais de 500 imagens provindas de diferentes perfis da prórpria rede social, que foram utilizadas durante o período de treinamento do sistema. Esta etapa consistiu em criar um filtro baseado na média dos rostos.

	Com o resultado do treinamento disponível, podemos então reconhecer uma nova foto e projetá-la sobre o filtro. Isto é importante para que ruídos, tais como iluminação e enquadramento na foto, sejam absorvidos e características relevantes da pessoa destacadas. Assim, podemos iniciar a busca e comparação com os demais usuários do Facebook, dado a rede limitada.

	Conquanto, exibiremos um grupo de usuários que possuem maior semelhança entre a sua foto de perfil e a foto utilizada na busca. Além disto, o resultado pode ou não ser satisfatório, isto é, encontrar ou não o usuário, pois pode ser que a pessoa buscada não possua foto ou que ela não tenha conta no Facebook.

\section{Solução proposta}

\section{Exprerimentos}

\section{Conclusões}

% ******************************************************
% 		REFERENCIAS BIBLIOGRÁFICAS
% ******************************************************
%\section{Referências}
\bibliographystyle{plain}
\begin{small}
  \bibliography{referencias}
\end{small}


\end{document}